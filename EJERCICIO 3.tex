\documentclass{article}
\usepackage[utf8]{inputenc}
\usepackage{amsmath}
\usepackage{amsthm}
\usepackage[english]{babel}


\theoremstyle{definition}
\newtheorem{theorem}{Theorem}[section]
\newtheorem{corollary}{Corollary}[theorem]
\newtheorem{lemma}[theorem]{Lemma}

\title{Demostraciones y Teoremas}
\author{JUAN CARLOS HUANCA MAMANI}

\begin{document}

\maketitle

\section{Teorema}

\begin{theorem}
    Supóngase que \( f \) es continua en \( a \), y \( f(a) > 0 \). Entonces existe un número \( \delta > 0 \) tal que \( f(x) > 0 \) para todo \( x \) que satisface \( |x - a| < \delta \). Análogamente, si \( f(a) < 0 \), entonces existe un número \( \delta > 0 \) tal que \( f(x) < 0 \) para todo \( x \) que satisface \( |x - a| < \delta \).
\end{theorem}

\noindent\textbf{DEMOSTRACIÓN.}

\begin{proof}
    Considérese el caso \( f(a) > 0 \), puesto que \( f \) es continua en \( a \), si \( \varepsilon > 0 \) existe un \( \delta > 0 \) tal que, para todo \( x \),
    
    \[\text{si} \quad |x - a| < \delta, \quad \text{entonces} \quad |f(x) - f(a)| < \varepsilon.\]
    
    Puesto que \( f(a) > 0 \), podemos tomar a \( f(a) \) como el \( \varepsilon \). Así, pues, existe \( \delta > 0 \) tal que para todo \( x \),
    
    \[ \text{si} \quad |x - a| < \delta, \quad \text{entonces} \quad |f(x) - f(a)| < f(a), \]
    
    y esta última igualdad implica \( f(x) > 0 \).
    
    Puede darse una demostración análoga en el caso \( f(a) < 0 \); tómese \( \varepsilon = -f(a) \). O también se puede aplicar el primer caso a la función \( -f \).
\end{proof}
\end{document}
